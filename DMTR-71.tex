\documentclass[DM,lsstdraft,STR,toc]{lsstdoc}
\usepackage{geometry}
\usepackage{longtable,booktabs}
\usepackage{enumitem}
\usepackage{arydshln}

\input meta.tex

\providecommand{\tightlist}{
  \setlength{\itemsep}{0pt}\setlength{\parskip}{0pt}}

\begin{document}

\def\milestoneName{2018 qserv large scale testing}
\def\milestoneId{LVV-P46}
\def\product{Distrib Database}

\setDocCompact{true}

\title[\milestoneId{}~Test Report]{\milestoneId{} (\milestoneName{})~Test Plan and Report}
\setDocRef{\lsstDocType-\lsstDocNum}
\setDocDate{\vcsdate}
\setDocUpstreamLocation{\url{https://github.com/lsst/lsst-texmf/examples}}
\author{ Fritz Mueller }

\input history_and_info.tex


\setDocAbstract{
This is the test plan and report for \milestoneId{} (\milestoneName{}), an LSST level 2 milestone pertaining to the Data Management Subsystem.
}


\maketitle

\section{Introduction}
\label{sect:intro}


\subsection{Objectives}
\label{sect:objectives}

Yearly functional and scale performance testing of the Qserv distributed
database system. ~Establishes Qserv's viability on growth curve toward
full production scale.



\subsection{System Overview}
\label{sect:systemoverview}

Qserv is a SQL-oriented MPP distributed database system built by LSST
for the purpose of hosting LSST catalog data products. ~Qserv is tested
yearly at large scale, on test datasets of ever-increasing size, to
ensure that development remains on a path toward delivering a system
that functions effectively at LSST release scales.\\
~\\

\hypertarget{applicable-documents}{%
\subsection{\texorpdfstring{Applicable Documents\\
}{Applicable Documents }}\label{applicable-documents}}

\citeds{LDM-555}: LSST Data Management Database Requirements\\
\citeds{LDM-135}: LSST Data Management Database Design\\
\citeds{LDM-552}: LSST Data Management Distributed Database Software Test
Specification


\subsection{Document Overview}
\label{sect:docoverview}

This document was generated from Jira, obtaining the relevant information from the 
\href{https://jira.lsstcorp.org/secure/Tests.jspa#/testPlan/LVV-P46}{LVV-P46}
~Jira Test Plan and related Test Cycles (
  \href{https://jira.lsstcorp.org/secure/Tests.jspa#/testCycle/LVV-C81}{LVV-C81}
).

Section \ref{sect:intro} provides an overview of the test campaign, the system under test (\product{}), the applicable documentation, and explains how this document is organized.
Section \ref{sect:configuration}  describes the configuration used for this test.
Section \ref{sect:personnel} describes the necessary roles and lists the individuals assigned to them.
%Section \ref{sect:plannedtestactivities} provides the list of planned test cycles and test cases, including all relevant information that fully describes the test campaign.

Section \ref{sect:overview} provides a summary of the test results, including an overview in Table \ref{table:summary}, an overall assessment statement and suggestions for possible improvements.
Section \ref{sect:detailedtestresults} provides detailed results for each step in each test case.

The current status of test plan LVV-P46 in Jira is Approved.

\subsection{References}
\label{sect:references}
\renewcommand{\refname}{}
\bibliography{lsst,refs,books,refs_ads}
\section{Test Configuration}
\label{sect:configuration}

\subsection{Data Collection}

  Observing is not required for this test campaign.

\subsection{Verification Environment}
\label{sect:hwconf}
  Qserv testing at scale requires a dedicated machine cluster:\\
~\\
- 25 to 50 "worker" nodes, each with ~on order 16 GB memory and on order
10 TB locally attached storage\\
- 1 to 2 "czar" nodes, minimally provisioned as above, but preferably
provisioned with more RAM and several TB of fast SSD storage\\
~\\
Suitable test clusters exist and have been used at both NCSA and
CC-IN2P3. ~Some testing at scale has also been conducted with
dynamically provisioned clusters on the Google cloud infrastructure.\\
~\\
A test dataset of appropriate size (per schedule in \citeds{LDM-552}) is also
required.





\section{Personnel}
\label{sect:personnel}

The following personnel are involved in this test activity:

\begin{itemize}
\item Test Plan (LVV-P46) owner: Fritz Mueller
\item Test Cycles:
\begin{itemize}
  \item LVV-C81 owner: 
    Fritz Mueller
  \begin{itemize}
    \item Test case \href{https://jira.lsstcorp.org/secure/Tests.jspa#/testCase/LVV-T1017}{LVV-T1017} tester: Fritz Mueller
    \item Test case \href{https://jira.lsstcorp.org/secure/Tests.jspa#/testCase/LVV-T1085}{LVV-T1085} tester: Fritz Mueller
    \item Test case \href{https://jira.lsstcorp.org/secure/Tests.jspa#/testCase/LVV-T1087}{LVV-T1087} tester: Fritz Mueller
    \item Test case \href{https://jira.lsstcorp.org/secure/Tests.jspa#/testCase/LVV-T1086}{LVV-T1086} tester: Fritz Mueller
    \item Test case \href{https://jira.lsstcorp.org/secure/Tests.jspa#/testCase/LVV-T1088}{LVV-T1088} tester: Fritz Mueller
    \item Test case \href{https://jira.lsstcorp.org/secure/Tests.jspa#/testCase/LVV-T1089}{LVV-T1089} tester: Fritz Mueller
    \item Test case \href{https://jira.lsstcorp.org/secure/Tests.jspa#/testCase/LVV-T1090}{LVV-T1090} tester: Fritz Mueller
  \end{itemize}
\end{itemize}
\item Additional Test Personnel involved: None
\end{itemize}

\newpage

\section{Overview of the Test Results}
\label{sect:overview}

\subsection{Summary}
\label{sect:summarytable}

\begin{longtable}{p{0.12\textwidth}p{0.2\textwidth}p{0.56\textwidth}p{0.12\textwidth}}
\toprule

  \multicolumn{3}{c}{ Test Cycle {\bf LVV-C81: 2018 Qserv Large Scale Testing
 }} \\\hline

  {\bf \footnotesize test case} & {\bf \footnotesize status} & {\bf \footnotesize comment} & {\bf \footnotesize issues} \\\toprule

    \href{https://jira.lsstcorp.org/secure/Tests.jspa#/testCase/LVV-T1017}{LVV-T1017}
    & Pass & Qserv docker containers installed without issue.
 &
    \\\hline
    \href{https://jira.lsstcorp.org/secure/Tests.jspa#/testCase/LVV-T1085}{LVV-T1085}
    & Pass & Short queries executed correctly in less than required times.
 &
    \\\hline
    \href{https://jira.lsstcorp.org/secure/Tests.jspa#/testCase/LVV-T1087}{LVV-T1087}
    & Pass & Queries executed correctly in less than required times.
 &
    \\\hline
    \href{https://jira.lsstcorp.org/secure/Tests.jspa#/testCase/LVV-T1086}{LVV-T1086}
    & Pass & Queries executed correctly in less than the required times.
 &
    \\\hline
    \href{https://jira.lsstcorp.org/secure/Tests.jspa#/testCase/LVV-T1088}{LVV-T1088}
    & Pass & Test results indicate the desired less-than-linear scaling rate for
scans of each type, within limits of machine resource exhaustion.
 &
    \\\hline
    \href{https://jira.lsstcorp.org/secure/Tests.jspa#/testCase/LVV-T1089}{LVV-T1089}
    & Conditional Pass & Load test excluded Object x (Source, ForcedSource) joins; see notes in
"Overall Assessment" section.
 &
    \\\hline
    \href{https://jira.lsstcorp.org/secure/Tests.jspa#/testCase/LVV-T1090}{LVV-T1090}
    & Conditional Pass & Heavy Load test excluded Object x (Source, ForcedSource) joins and was
limited by maximum shared-scan load that could be accommodated by the
current test tooling; see notes in "Overall Assessment" section.
 &
    \\\hline

\caption{Test Results Summary}
\label{table:summary}
\end{longtable}

\subsection{Overall Assessment}
\label{sect:overallassessment}

A performance problem was observed with execution of Object x Source and
Object x ForcedSource joins during this testing. ~While these joins
performed per expectation in isolation, and full-table-scans performed
per expectation in isolation, the combination of both under high loads
resulted in significantly degraded performance.\\
~\\
The development team spent a good deal of time investigating this
regression. ~It was found by running against older codebases that the
regression was independent of changes to either the Qserv codebase or
the underlying MariaDB database engine. ~Indeed, when historic code and
datasets were run on the same hardware systems which had been used for
performance testing previously, at the same scale as had been run
previously, this same join+scans performance degradation was evident.\\
~\\
The development team has concluded that changes to the operating
environment (OS kernel, firmware) must be implicated, since these are
the only remaining variables which could not be rolled back in the
efforts to duplicate previous performance measurements. ~Investigations
continue, and the issue is expected to be resolved before the next round
of large scale testing.\\
~\\
Additionally, the heavy load test taxed the existing test tooling and
single-master configuration their limits, due to increased result set
sized, overheads from 50\% more scans, limits on simultaneous database
connections from the test script, etc. ~For the next round of testing, a
more robust test harness will need to be developed, and multiple head
nodes will be required to accommodate the increased result-handling
loads.\\
~\\
In the meantime, concurrent joins were disabled during the load tests,
and the existing tooling was run to as high a load as possible for the
heavy load test, in order to gather/document as much meaningful
performance assessment as possible in the current operating environment.
~Since indications are that the Qserv codebase itself is not implicated
in the join performance regression, and Qserv otherwise appears to be
healthy and performing per design, the load tests have been marked with
a "conditional pass".


\subsection{Recommended Improvements}
\label{sect:recommendations}

\begin{itemize}
\tightlist
\item
  Test tooling to be revamped to support higher concurrency in database
  connections and to gracefully handle larger cumulative result sizes.
\item
  Subsequent test regimes will require more than a single master node.
\end{itemize}


\newpage
\section{Detailed Test Results}
\label{sect:detailedtestresults}


  \subsection{Test Cycle LVV-C81 }

Open test cycle {\it \href{https://jira.lsstcorp.org/secure/Tests.jspa#/testrun/LVV-C81}{2018 Qserv Large Scale Testing
}} in Jira.

  2018 Qserv Large Scale Testing
\\
  Status: Done

  This test cycle establishes that:\\

\begin{enumerate}
\tightlist
\item
  Qserv functional query requirements are met,~
\item
  Qserv's shared scan infrastructure performs per design, and
\item
  Qserv meets query response requirements under load, with data at scale
  of 30\% DR1 data volume.
\end{enumerate}


  \subsubsection{Software Version/Baseline}
    Qserv built from git SHA 06cdeda75 (published to docker hub as
qserv/qserv:travis\_DM-13961)


  \subsubsection{Configuration}
    \textbf{Hardware}\\

\begin{itemize}
\tightlist
\item
  50 nodes:

  \begin{itemize}
  \tightlist
  \item
    DELL PowerEdge R620 (Dell Spec Sheet) for nodes 1-25~
  \item
    DELL PowerEdge R630 (Dell Spec Sheet) for nodes 26-50
  \end{itemize}
\item
  2 x Processors Intel Xeon E5-2603v2 @ 1.80 Ghz 4 core
\item
  10 MB cache, 6.4 GT/s, 80W
\item
  Memory 16 GB DDR-3 @ 1600MHz (2x8GB)
\item
  2 x hard drive 250GB SATA 7200 Rpm 2,5'' hotplug (OS)
\item
  8 x hard drive 1 TB Nearline SAS 6 Gbps 7200 Rpm 2,5'' hotplug (DATA)
\item
  1 x card RAID H710p with 1 GB nvram
\item
  1 x card1 GbE 4 ports Broadcom® 5720 Base-T
\item
  1 x card iDRAC 7 Enterprise
\end{itemize}

\textbf{Dataset Information\\
}~\\

\begin{longtable}[]{@{}llll@{}}
\toprule
Table & Row Count & .MYD size {[}TB{]} & .MYI size
{[}TB{]}\tabularnewline
\midrule
\endhead
Object & 5,662,102,056 & 6.86 & 0.15\tabularnewline
Source & 104,440,271,322 & 49.4 & 5.7\tabularnewline
ForcedSource & 515,549,769,246 & 16.4 & 12.9\tabularnewline
\bottomrule
\end{longtable}

Total MySQL data dir size: 93.6 TB\\
~\\
DR1 numbers are available in Document-16168 under ``Data Releases''\\
~\\
Object, Source and ForcedSource are at slightly less than
\textasciitilde{}30\% of DR1 level due to some empty chunks generated
erroneously during the duplication phase. This difference is marginal
and will not affect test results.


  \subsubsection{Test Cases in LVV-C81 Test Cycle}


    \paragraph{Test Case LVV-T1017 - Qserv Preparation
 }\mbox{}\\

Open  \href{https://jira.lsstcorp.org/secure/Tests.jspa#/testCase/LVV-T1017}{\textit{ LVV-T1017 } }
test case in Jira.

    Before running any of the performance test cases, Qserv must be
installed on an appropriate test cluster (e.g. the test machine cluster
at CC-IN2P3). ~To upgrade Qserv software on the cluster in preparation
for testing, follow directions at
http://www.slac.stanford.edu/exp/lsst/qserv/2015\_10/HOW-TO/cluster-deployment.html.\\
~\\
The performance tests will also require an appropriately sized test
dataset to be synthesized and ingested, per the yearly dataset sizing
schedule described in \citeds{LDM-552}, section 2.2.1. ~Tools for synthesis of
ingest of test datasets may be found in the LSST GitHub repot at
https://github.com/lsst-dm/db\_tests\_kpm*. ~Detailed use and context
information for the tools is described in
https://jira.lsstcorp.org/browse/DM-8405.\\
~\\
It has also been found that the Qserv shard servers must have
engine-independent statistics loaded for the larger tables in the test
dataset, and be properly configured so that the MariaDB query planner
can make use of those statistics. ~More information on this issue is
available at
https://confluence.lsstcorp.org/pages/viewpage.action?pageId=58950786.


    \textbf{ Preconditions}:\\
    

    Execution status: {\bf Pass }

    Final comment:\\Qserv docker containers installed without issue.



    Detailed step results:

    \begin{longtable}{p{1cm}p{2cm}p{13cm}}
    \hline
    {Step} & \multicolumn{2}{c}{Description, Results and Status}\\ \hline
      1 & Description &

      \begin{minipage}[t]{13cm}{\footnotesize
      Install/upgrade Qserv on a test cluster, following directions at
http://www.slac.stanford.edu/exp/lsst/qserv/2015\_10/HOW-TO/cluster-deployment.html

      \vspace{\dp0}
      } \end{minipage} \\
      \\ \cdashline{2-3}

      & Expected Result & 

      \begin{minipage}[t]{13cm}{\footnotesize
      Qserv installed

      \vspace{\dp0}
      } \end{minipage} \\
      \\ \cdashline{2-3}

      & \begin{minipage}[t]{2cm}{Actual\\ Result}\end{minipage}   & 
      \begin{minipage}[t]{13cm}{\footnotesize
      Qserv installed and testing performed at CC-IN2P3, on nodes ccqserv100 -
ccqserv124.

      \vspace{\dp0}
      } \end{minipage} \\
      \\ \cdashline{2-3}


      & Status          & Pass \\ \hline

      2 & Description &

      \begin{minipage}[t]{13cm}{\footnotesize
      Synthesize and load and appropriately sized test dataset per the yearly
dataset sizing schedule described in LDM-552, section 2.2.1. Tools for
synthesis of ingest of test datasets may be found in the LSST GitHub
repot at https://github.com/lsst dm/db\_tests\_kpm*. ~Detailed use and
context information for the tools is described in
https://jira.lsstcorp.org/browse/DM-8405.\\
~\\

      \vspace{\dp0}
      } \end{minipage} \\
      \\ \cdashline{2-3}

      & Expected Result & 

      \begin{minipage}[t]{13cm}{\footnotesize
      Test dataset loaded

      \vspace{\dp0}
      } \end{minipage} \\
      \\ \cdashline{2-3}

      & \begin{minipage}[t]{2cm}{Actual\\ Result}\end{minipage}   & 
      \begin{minipage}[t]{13cm}{\footnotesize
      Test dataset loaded as database LSST30.

      \vspace{\dp0}
      } \end{minipage} \\
      \\ \cdashline{2-3}


      & Status          & Pass \\ \hline

    \end{longtable}


    \paragraph{Test Case LVV-T1085 - Short Queries Functional Test
 }\mbox{}\\

Open  \href{https://jira.lsstcorp.org/secure/Tests.jspa#/testCase/LVV-T1085}{\textit{ LVV-T1085 } }
test case in Jira.

    The objective of this test is to ensure that the short queries are
performing as expected and establish a timing baseline benchmark for
these types of queries.


    \textbf{ Preconditions}:\\
    QSERV has been set-up following procedure at ~LVV-T1017.


    Execution status: {\bf Pass }

    Final comment:\\Short queries executed correctly in less than required times.



    Detailed step results:

    \begin{longtable}{p{1cm}p{2cm}p{13cm}}
    \hline
    {Step} & \multicolumn{2}{c}{Description, Results and Status}\\ \hline
      1 & Description &

      \begin{minipage}[t]{13cm}{\footnotesize
      Execute single object selection:\\
~\\
\textbf{SELECT} * \textbf{FROM} Object~\textbf{WHERE} deepSourceId =
9292041530376264\\
~\\
and record execution time.

      \vspace{\dp0}
      } \end{minipage} \\
      \\ \cdashline{2-3}

      & Expected Result & 

      \begin{minipage}[t]{13cm}{\footnotesize
      Query runs in less than 10 seconds.

      \vspace{\dp0}
      } \end{minipage} \\
      \\ \cdashline{2-3}

      & \begin{minipage}[t]{2cm}{Actual\\ Result}\end{minipage}   & 
      \begin{minipage}[t]{13cm}{\footnotesize
      Execution time 0.23 sec.

      \vspace{\dp0}
      } \end{minipage} \\
      \\ \cdashline{2-3}


      & Status          & Pass \\ \hline

      2 & Description &

      \begin{minipage}[t]{13cm}{\footnotesize
      Execute spatial area selection from Object:\\
~\\
\textbf{SELECT COUNT(*)} \textbf{FROM} Object \textbf{WHERE}~\\

~qserv\_areaspec\_box(316.582327, −6.839078, 316.653938, −6.781822)

and record execution time.

      \vspace{\dp0}
      } \end{minipage} \\
      \\ \cdashline{2-3}

      & Expected Result & 

      \begin{minipage}[t]{13cm}{\footnotesize
      Query runs in less than 10 seconds.

      \vspace{\dp0}
      } \end{minipage} \\
      \\ \cdashline{2-3}

      & \begin{minipage}[t]{2cm}{Actual\\ Result}\end{minipage}   & 
      \begin{minipage}[t]{13cm}{\footnotesize
      Execution time 3.34 sec.

      \vspace{\dp0}
      } \end{minipage} \\
      \\ \cdashline{2-3}


      & Status          & Pass \\ \hline

    \end{longtable}


    \paragraph{Test Case LVV-T1087 - Full Table Joins Functional Test
 }\mbox{}\\

Open  \href{https://jira.lsstcorp.org/secure/Tests.jspa#/testCase/LVV-T1087}{\textit{ LVV-T1087 } }
test case in Jira.

    The objective of this test is to ensure that the full table join queries
are performing as expected and establish a timing baseline benchmark for
these types of queries.


    \textbf{ Preconditions}:\\
    QSERV has been set-up following procedure at ~LVV-T1017.


    Execution status: {\bf Pass }

    Final comment:\\Queries executed correctly in less than required times.



    Detailed step results:

    \begin{longtable}{p{1cm}p{2cm}p{13cm}}
    \hline
    {Step} & \multicolumn{2}{c}{Description, Results and Status}\\ \hline
      1 & Description &

      \begin{minipage}[t]{13cm}{\footnotesize
      Execute query:\\
~\\
\textbf{SELECT} o.deepSourceId, s.objectId, s.id, o.ra, o.decl\\
\textbf{~ ~ FROM} Object o, Source s WHERE o.deepSourceId=s.objectId\\
\hspace*{0.333em} ~ \textbf{AND} s . flux\_sinc \textbf{BETWEEN} 0.3
\textbf{AND} 0.31\\
~\\
and record execution time.

      \vspace{\dp0}
      } \end{minipage} \\
      \\ \cdashline{2-3}

      & Expected Result & 

      \begin{minipage}[t]{13cm}{\footnotesize
      Query expected to run in less than 12 hours.

      \vspace{\dp0}
      } \end{minipage} \\
      \\ \cdashline{2-3}

      & \begin{minipage}[t]{2cm}{Actual\\ Result}\end{minipage}   & 
      \begin{minipage}[t]{13cm}{\footnotesize
      Query executed in \textasciitilde{}112 min.

      \vspace{\dp0}
      } \end{minipage} \\
      \\ \cdashline{2-3}


      & Status          & Pass \\ \hline

      2 & Description &

      \begin{minipage}[t]{13cm}{\footnotesize
      Execute query:\\
~\\
\textbf{SELECT} o.deepSourceId, f.psfFlux \textbf{FROM} Object o,
ForcedSource f\\
\textbf{~ ~ WHERE} o.deepSourceId=f.deepSourceId\\
\textbf{~ ~ AND} f . psfFlux \textbf{BETWEEN} 0.13 \textbf{AND} 0.14\\
~\\
and record execution time.

      \vspace{\dp0}
      } \end{minipage} \\
      \\ \cdashline{2-3}

      & Expected Result & 

      \begin{minipage}[t]{13cm}{\footnotesize
      Query expected to run in less than 12 hours.

      \vspace{\dp0}
      } \end{minipage} \\
      \\ \cdashline{2-3}

      & \begin{minipage}[t]{2cm}{Actual\\ Result}\end{minipage}   & 
      \begin{minipage}[t]{13cm}{\footnotesize
      Query executed in \textasciitilde{}5 hr.

      \vspace{\dp0}
      } \end{minipage} \\
      \\ \cdashline{2-3}


      & Status          & Pass \\ \hline

    \end{longtable}


    \paragraph{Test Case LVV-T1086 - Full Table Scans Functional Test
 }\mbox{}\\

Open  \href{https://jira.lsstcorp.org/secure/Tests.jspa#/testCase/LVV-T1086}{\textit{ LVV-T1086 } }
test case in Jira.

    The objective of this test is to ensure that the full table scan queries
are performing as expected and establish a timing baseline benchmark for
these types of queries.


    \textbf{ Preconditions}:\\
    QSERV has been set-up following procedure at ~LVV-T1017.


    Execution status: {\bf Pass }

    Final comment:\\Queries executed correctly in less than the required times.



    Detailed step results:

    \begin{longtable}{p{1cm}p{2cm}p{13cm}}
    \hline
    {Step} & \multicolumn{2}{c}{Description, Results and Status}\\ \hline
      1 & Description &

      \begin{minipage}[t]{13cm}{\footnotesize
      Execute query:\\
~\\
\textbf{SELECT} ra , decl , u\_psfFlux , g\_psfFlux , r\_psfFlux
\textbf{FROM} Object\\
\textbf{WHERE} y\_shapeIxx \textbf{BETWEEN} 20 \textbf{AND} 20.1\\
~\\
~\\
and record execution time and output size.

      \vspace{\dp0}
      } \end{minipage} \\
      \\ \cdashline{2-3}

      & Expected Result & 

      \begin{minipage}[t]{13cm}{\footnotesize
      Query expected to run in less than 1 hour.\\
~\\

      \vspace{\dp0}
      } \end{minipage} \\
      \\ \cdashline{2-3}

      & \begin{minipage}[t]{2cm}{Actual\\ Result}\end{minipage}   & 
      \begin{minipage}[t]{13cm}{\footnotesize
      Query executed in \textasciitilde{}20 min, 83MB output.

      \vspace{\dp0}
      } \end{minipage} \\
      \\ \cdashline{2-3}


      & Status          & Pass \\ \hline

      2 & Description &

      \begin{minipage}[t]{13cm}{\footnotesize
      Execute query:\\
~\\
\textbf{SELECT} COUNT(*) \textbf{FROM} Source \textbf{WHERE} flux\_sinc
\textbf{BETWEEN} 1 \textbf{AND} 1.1\\
~\\
and record the execution time

      \vspace{\dp0}
      } \end{minipage} \\
      \\ \cdashline{2-3}

      & Expected Result & 

      \begin{minipage}[t]{13cm}{\footnotesize
      Query expected to run in less than 12 hours.

      \vspace{\dp0}
      } \end{minipage} \\
      \\ \cdashline{2-3}

      & \begin{minipage}[t]{2cm}{Actual\\ Result}\end{minipage}   & 
      \begin{minipage}[t]{13cm}{\footnotesize
      Query executed in \textasciitilde{}104 min.

      \vspace{\dp0}
      } \end{minipage} \\
      \\ \cdashline{2-3}


      & Status          & Pass \\ \hline

      3 & Description &

      \begin{minipage}[t]{13cm}{\footnotesize
      Execute query:\\
~\\
\textbf{SELECT} COUNT(*) \textbf{FROM} ForcedSource \textbf{WHERE}
psfFlux \textbf{BETWEEN} 0.1 \textbf{AND} 0.2\\
~\\
and record the execution time

      \vspace{\dp0}
      } \end{minipage} \\
      \\ \cdashline{2-3}

      & Expected Result & 

      \begin{minipage}[t]{13cm}{\footnotesize
      Query expected to run in less than 12 hours.

      \vspace{\dp0}
      } \end{minipage} \\
      \\ \cdashline{2-3}

      & \begin{minipage}[t]{2cm}{Actual\\ Result}\end{minipage}   & 
      \begin{minipage}[t]{13cm}{\footnotesize
      Query executed in \textasciitilde{}48 min.

      \vspace{\dp0}
      } \end{minipage} \\
      \\ \cdashline{2-3}


      & Status          & Pass \\ \hline

    \end{longtable}


    \paragraph{Test Case LVV-T1088 - Concurrent Scans Scaling Test
 }\mbox{}\\

Open  \href{https://jira.lsstcorp.org/secure/Tests.jspa#/testCase/LVV-T1088}{\textit{ LVV-T1088 } }
test case in Jira.

    This test will show that average completion-time of full-scan queries of
the Object catalog table grows sub-linearly with respect to the number
of simultaneously active full-scan queries, within the limits of machine
resource exhaustion.


    \textbf{ Preconditions}:\\
    \begin{enumerate}
\tightlist
\item
  A test catalog of appropriate size (see schedule detail in \citeds{LDM-552},
  section 2.2.1), prepared and ingested into the Qserv instance under
  test as detailed in LVV-T1017.
\item
  The concurrency load execution script, runQueries.py, maintained in
  the LSST Qserv github repository here:
  https://github.com/lsst/qserv/blob/master/admin/tools/docker/deployment/in2p3/runQueries.py
\end{enumerate}


    Execution status: {\bf Pass }

    Final comment:\\Test results indicate the desired less-than-linear scaling rate for
scans of each type, within limits of machine resource exhaustion.



    Detailed step results:

    \begin{longtable}{p{1cm}p{2cm}p{13cm}}
    \hline
    {Step} & \multicolumn{2}{c}{Description, Results and Status}\\ \hline
      1 & Description &

      \begin{minipage}[t]{13cm}{\footnotesize
      Repeat steps 2 through 5 below, where "pool of interest" is taken first
to be "FTSObj" and subsequently "FTSSrc":

      \vspace{\dp0}
      } \end{minipage} \\
      \\ \cdashline{2-3}

      & Expected Result & 

      \begin{minipage}[t]{13cm}{\footnotesize
      At end of each pass, a graph indicating scan scaling rate and machine
resource exhaustion cutoff.

      \vspace{\dp0}
      } \end{minipage} \\
      \\ \cdashline{2-3}

      & \begin{minipage}[t]{2cm}{Actual\\ Result}\end{minipage}   & 
      \begin{minipage}[t]{13cm}{\footnotesize
      Object\\

\begin{itemize}
\tightlist
\item
  2 scans: \textasciitilde{}20 min
\item
  5 scans: \textasciitilde{}20 min
\item
  10 scans: \textasciitilde{}22 min
\item
  20 scans: \textasciitilde{}25 min
\item
  40 scans: \textasciitilde{}72 min (machine resource exhaustion)
\end{itemize}

ForcedSource:

\begin{itemize}
\tightlist
\item
  2 scans: \textasciitilde{} 51 min~
\item
  4 scans: \textasciitilde{} 61 min
\item
  10 scans: \textasciitilde{}270 min (machine resource exhaustion)
\end{itemize}

      \vspace{\dp0}
      } \end{minipage} \\
      \\ \cdashline{2-3}


      & Status          & Pass \\ \hline

      2 & Description &

      \begin{minipage}[t]{13cm}{\footnotesize
      Inspect and modify the CONCURRENCY and TARGET\_RATES dictionaries in the
runQueries.py script. Set CONCURRENCY initially to 1 for the query pool
of interest, and to 0 for all other query pools. Set TARGET\_RATES for
the query pool of interest to the yearly value per table in LDM-552,
section 2.2.1.

      \vspace{\dp0}
      } \end{minipage} \\
      \\ \cdashline{2-3}

      & Expected Result & 

      \begin{minipage}[t]{13cm}{\footnotesize
      rueQueries.py script updated with appropriate values for test iteration

      \vspace{\dp0}
      } \end{minipage} \\
      \\ \cdashline{2-3}

      & \begin{minipage}[t]{2cm}{Actual\\ Result}\end{minipage}   & 
      \begin{minipage}[t]{13cm}{\footnotesize
      Appropriate edits made.

      \vspace{\dp0}
      } \end{minipage} \\
      \\ \cdashline{2-3}


      & Status          & Pass \\ \hline

      3 & Description &

      \begin{minipage}[t]{13cm}{\footnotesize
      Execute the runQueries.py script and let it run for at least one, but
preferably several, query cycles.

      \vspace{\dp0}
      } \end{minipage} \\
      \\ \cdashline{2-3}

      & Expected Result & 

      \begin{minipage}[t]{13cm}{\footnotesize
      Test script executes producing log file.

      \vspace{\dp0}
      } \end{minipage} \\
      \\ \cdashline{2-3}

      & \begin{minipage}[t]{2cm}{Actual\\ Result}\end{minipage}   & 
      \begin{minipage}[t]{13cm}{\footnotesize
      Script executed per design.

      \vspace{\dp0}
      } \end{minipage} \\
      \\ \cdashline{2-3}


      & Status          & Pass \\ \hline

      4 & Description &

      \begin{minipage}[t]{13cm}{\footnotesize
      Examine log file output and compile performance statistics to obtain a
growth curve point for the pool of interest for the test report.

      \vspace{\dp0}
      } \end{minipage} \\
      \\ \cdashline{2-3}

      & Expected Result & 

      \begin{minipage}[t]{13cm}{\footnotesize
      Logs indicate either successful test run, providing another growth point
for curve, or errors indicating machine resource exhaustion cutoff has
been reached.

      \vspace{\dp0}
      } \end{minipage} \\
      \\ \cdashline{2-3}

      & \begin{minipage}[t]{2cm}{Actual\\ Result}\end{minipage}   & 
      \begin{minipage}[t]{13cm}{\footnotesize
      Script executions and log contents as expected.

      \vspace{\dp0}
      } \end{minipage} \\
      \\ \cdashline{2-3}


      & Status          & Pass \\ \hline

      5 & Description &

      \begin{minipage}[t]{13cm}{\footnotesize
      Adjust the CONCURRENCY value for the pool of interest and repeat from
step 3 to establish the growth trend and machine resource exhaustion
cutoff for the query pool of interest to an acceptable degree of
accuracy.

      \vspace{\dp0}
      } \end{minipage} \\
      \\ \cdashline{2-3}

      & Expected Result & 

      \begin{minipage}[t]{13cm}{\footnotesize
      Average query execution time for full scan queries of each class should
be demonstrated to grow sub-linearly in the number of concurrent queries
to the limits of machine resource exhaustion.

      \vspace{\dp0}
      } \end{minipage} \\
      \\ \cdashline{2-3}

      & \begin{minipage}[t]{2cm}{Actual\\ Result}\end{minipage}   & 
      \begin{minipage}[t]{13cm}{\footnotesize
      Test indicated less-than-linear scaling rate for scans of each type,
within limits of machine resource exhaustion.

      \vspace{\dp0}
      } \end{minipage} \\
      \\ \cdashline{2-3}


      & Status          & Pass \\ \hline

    \end{longtable}


    \paragraph{Test Case LVV-T1089 - Load Test
 }\mbox{}\\

Open  \href{https://jira.lsstcorp.org/secure/Tests.jspa#/testCase/LVV-T1089}{\textit{ LVV-T1089 } }
test case in Jira.

    This test will check that Qserv is able to meet average query completion
time targets per query class under a representative load of simultaneous
high and low volume queries while running against an appropriately
scaled test catalog.


    \textbf{ Preconditions}:\\
    QSERV has been set-up following procedure at ~LVV-T1017


    Execution status: {\bf Conditional Pass }

    Final comment:\\Load test excluded Object x (Source, ForcedSource) joins; see notes in
"Overall Assessment" section.



    Detailed step results:

    \begin{longtable}{p{1cm}p{2cm}p{13cm}}
    \hline
    {Step} & \multicolumn{2}{c}{Description, Results and Status}\\ \hline
      1 & Description &

      \begin{minipage}[t]{13cm}{\footnotesize
      Inspect and modify the CONCURRENCY and TARGET\_RATES dictionaries in the
runQueries.py script. ~Set CONCURRENCY and TARGET\_RATES for all pools
to the yearly value per table in LDM-552, section 2.2.1.

      \vspace{\dp0}
      } \end{minipage} \\
      \\ \cdashline{2-3}

      & Expected Result & 

      \begin{minipage}[t]{13cm}{\footnotesize
      Script updated with appropriate values.

      \vspace{\dp0}
      } \end{minipage} \\
      \\ \cdashline{2-3}

      & \begin{minipage}[t]{2cm}{Actual\\ Result}\end{minipage}   & 
      \begin{minipage}[t]{13cm}{\footnotesize
      Script updated without difficulty.

      \vspace{\dp0}
      } \end{minipage} \\
      \\ \cdashline{2-3}


      & Status          & Pass \\ \hline

      2 & Description &

      \begin{minipage}[t]{13cm}{\footnotesize
      Execute the runQueries.py script and let it run for 24 hours.

      \vspace{\dp0}
      } \end{minipage} \\
      \\ \cdashline{2-3}

      & Expected Result & 

      \begin{minipage}[t]{13cm}{\footnotesize
      Script runs without error and produces output log.

      \vspace{\dp0}
      } \end{minipage} \\
      \\ \cdashline{2-3}

      & \begin{minipage}[t]{2cm}{Actual\\ Result}\end{minipage}   & 
      \begin{minipage}[t]{13cm}{\footnotesize
      Script ran and log files were generated per expectation.

      \vspace{\dp0}
      } \end{minipage} \\
      \\ \cdashline{2-3}


      & Status          & Pass \\ \hline

      3 & Description &

      \begin{minipage}[t]{13cm}{\footnotesize
      Examine log file output and compile average query execution times per
query type; and compare to yearly target values per table in LDM-552,
section 2.2.1.

      \vspace{\dp0}
      } \end{minipage} \\
      \\ \cdashline{2-3}

      & Expected Result & 

      \begin{minipage}[t]{13cm}{\footnotesize
      Average query times per query type equal or less than corresponding
yearly target values in LDM-552, section 2.2.1.

      \vspace{\dp0}
      } \end{minipage} \\
      \\ \cdashline{2-3}

      & \begin{minipage}[t]{2cm}{Actual\\ Result}\end{minipage}   & 
      \begin{minipage}[t]{13cm}{\footnotesize
      Query through-put over 24 hours:\\

\begin{itemize}
\tightlist
\item
  625,245 Low Volume queries finished --- Baseline: 604,800
\item
  201 Object scans --- Baseline: 192
\item
  4 Source scans --- Baseline: 4
\item
  4 ForcedSource scans --- Baseline: 4
\item
  0 Object-Source joins (see "Overall Assessment" section) --- Baseline:
  8~
\item
  0 Object-ForcedSource joins (see "Overall Assessment" section) ---
  Baseline: 4~
\item
  53 NearNeighbor queries --- Baseline: 48
\end{itemize}

Average query times:

\begin{itemize}
\tightlist
\item
  Low Volume queries 2.92 sec/query --- Baseline: under 10 sec.
\item
  Object scans 44.5 min/query --- Baseline: under 1 hour
\item
  Source scans 5.4 hr/query --- Baseline: under 12 hours
\item
  ForcedSource scans 5.4 hr/query --- Baseline: under 12 hours
\item
  Object-Source joins (not measured; see "Overall Assessment" section)
  --- Baseline: under 12 hours
\item
  Object-ForcedSource joins (not measured; see "Overall Assessment"
  section) --- Baseline: under 12 hours
\item
  NearNeighbor queries 38.9 min/query --- Baseline: under 12 hours
\end{itemize}

      \vspace{\dp0}
      } \end{minipage} \\
      \\ \cdashline{2-3}


      & Status          & Conditional Pass \\ \hline

    \end{longtable}


    \paragraph{Test Case LVV-T1090 - Heavy Load Test
 }\mbox{}\\

Open  \href{https://jira.lsstcorp.org/secure/Tests.jspa#/testCase/LVV-T1090}{\textit{ LVV-T1090 } }
test case in Jira.

    This test will check that Qserv is able to meet average query completion
time targets per query class under a higher than average load of
simultaneous high and low volume queries while running against an
appropriately scaled test catalog.


    \textbf{ Preconditions}:\\
    QSERV has been set-up following procedure at ~LVV-T1017


    Execution status: {\bf Conditional Pass }

    Final comment:\\Heavy Load test excluded Object x (Source, ForcedSource) joins and was
limited by maximum shared-scan load that could be accommodated by the
current test tooling; see notes in "Overall Assessment" section.



    Detailed step results:

    \begin{longtable}{p{1cm}p{2cm}p{13cm}}
    \hline
    {Step} & \multicolumn{2}{c}{Description, Results and Status}\\ \hline
      1 & Description &

      \begin{minipage}[t]{13cm}{\footnotesize
      Inspect and modify the CONCURRENCY and TARGET\_RATES dictionaries in the
runQueries.py script. ~Set CONCURRENCY and TARGET\_RATES for LV query
pool to 2020 value per table in LDM-552, section 2.2.1.~ Set CONCURRENCY
and TARGET\_RATES for all other query pools to values in next column
over from current year column (or to 2020 values +10\% if year is 2020)
per table in LDM-552, section 2.2.1.

      \vspace{\dp0}
      } \end{minipage} \\
      \\ \cdashline{2-3}

      & Expected Result & 

      \begin{minipage}[t]{13cm}{\footnotesize
      Script updated with appropriate values.

      \vspace{\dp0}
      } \end{minipage} \\
      \\ \cdashline{2-3}

      & \begin{minipage}[t]{2cm}{Actual\\ Result}\end{minipage}   & 
      \begin{minipage}[t]{13cm}{\footnotesize
      Script updated without difficulty.

      \vspace{\dp0}
      } \end{minipage} \\
      \\ \cdashline{2-3}


      & Status          & Pass \\ \hline

      2 & Description &

      \begin{minipage}[t]{13cm}{\footnotesize
      Execute the runQueries.py script and let it run for 24 hrs.

      \vspace{\dp0}
      } \end{minipage} \\
      \\ \cdashline{2-3}

      & Expected Result & 

      \begin{minipage}[t]{13cm}{\footnotesize
      Script runs without error and produces output log.

      \vspace{\dp0}
      } \end{minipage} \\
      \\ \cdashline{2-3}

      & \begin{minipage}[t]{2cm}{Actual\\ Result}\end{minipage}   & 
      \begin{minipage}[t]{13cm}{\footnotesize
      Script ran, but did not finish expected number of scans due to
proxy/thread crashes. ~This appears to be a limitation of the test
script and the mysqlproxy front end.

      \vspace{\dp0}
      } \end{minipage} \\
      \\ \cdashline{2-3}


      & Status          & Conditional Pass \\ \hline

      3 & Description &

      \begin{minipage}[t]{13cm}{\footnotesize
      Examine log file output and compile average query execution times per
query type.

      \vspace{\dp0}
      } \end{minipage} \\
      \\ \cdashline{2-3}

      & Expected Result & 

      \begin{minipage}[t]{13cm}{\footnotesize
      Average query times per query type equal or less than corresponding
yearly target values in LDM-552, section 2.2.1.

      \vspace{\dp0}
      } \end{minipage} \\
      \\ \cdashline{2-3}

      & \begin{minipage}[t]{2cm}{Actual\\ Result}\end{minipage}   & 
      \begin{minipage}[t]{13cm}{\footnotesize
      Query through-put over 24 hours:\\

\begin{itemize}
\tightlist
\item
  779,308 Low Volume queries ~--- Baseline: 691,200
\item
  188 Object scans --- Baseline: 288
\item
  4 Source scans --- Baseline: 6
\item
  4 ForcedSource scans --- Baseline: 6
\item
  0 Object-Source joins ~--- Baseline: 12~
\item
  0 Object-ForcedSource joins ~--- Baseline: 6
\item
  53 NearNeighbor queries ~--- Baseline: 72
\end{itemize}

Average query times:\\

\begin{itemize}
\tightlist
\item
  Low Volume queries 3.67 sec/query --- Baseline: under 10 sec
\item
  Object scans 44 min/query ~--- Baseline: under 1 hour
\item
  Source scans 4.8 hr/query --- Baseline: under 12 hours
\item
  ForcedSource scans 4.8 hr/query --- Baseline: under 12 hours
\item
  Object-Source joins (not measured; see "Overall Assessment" section)
  ~--- Baseline: under 12 hours
\item
  Object-ForcedSource joins (not measured; see "Overall Assessment"
  section) --- Baseline: under 12 hours
\item
  NearNeighbor queries 39 min/query --- Baseline: under 12 hours
\end{itemize}

      \vspace{\dp0}
      } \end{minipage} \\
      \\ \cdashline{2-3}


      & Status          & Conditional Pass \\ \hline

    \end{longtable}


\input{appendix.tex}
\end{document}
